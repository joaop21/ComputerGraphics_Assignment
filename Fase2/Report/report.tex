\documentclass[a4paper]{article}

\usepackage[utf8]{inputenc}
\usepackage[portuges]{babel}
\usepackage{indentfirst}
\usepackage{graphicx}
\usepackage{float}
\usepackage{caption}
\usepackage{subcaption}
\usepackage[T1]{fontenc}
\usepackage{listings}
\usepackage{amsmath}
\usepackage{mathtools}
\renewcommand{\familydefault}{\sfdefault}


\title{Projeto de Computação Gráfica - Fase 2}
\author{Diogo Braga A82547 \and João Silva A82005 \and Ricardo Caçador A81064
\and Ricardo Veloso A81919}
\date{\today}

\begin{document}

\maketitle

\begin{abstract}
Neste relatório é apresentada a segunda fase dum projeto no qual a intenção é desenvolver um mecanismo baseado em gráficos 3D e fornecer exemplos de uso que mostrem o seu potencial. Este projeto é desenvolvido no âmbito da unidade curricular de Computação Gráfica.
\end{abstract}

\tableofcontents


\newpage

\section{Introdução}
\label{sec:intro}

Esta segunda fase tem como objetivo a realização de algumas etapas, nomeadamente: 
\begin{enumerate}  
\item A criação dum parser \emph{XML} mais conciso;
\item A alteração/evolução da estrutura dos ficheiros \emph{XML};
\item A criação de estruturas capazes de armazenar as figuras necessárias;
\item A criação de funções que leiam as estruturas e as desenhem;
\item A criação dum modelo estático do sistema solar.
\end{enumerate}

De seguida iremos apresentar todos os algoritmos e decisões relativos à realização destas etapas, bem como as respetivas explicações de cada passo. 
Serão ainda apresentadas figuras e esquemas que ilustrem os passos do processo desenvolvido.



\section{Estrutura da Pasta do Projeto}
\label{sec:estrutura}

Para um entendimento mais claro da estrutura do projeto, achamos por bem referenciar a estrutura da pasta do projeto.
O projeto entregue contêm para além do relatório, 4 pastas como é possível verificar na seguinte figura.

\begin{figure}[H]
\centering
\includegraphics[scale=1.0]{estrutura.png}
\caption{Estrutura da pasta.}
\label{img:estrutura}
\end{figure}

Na pasta \textbf{Engine} residem os ficheiros relativos ao programa \emph{engine}, bem como as bibliotecas (.h) criadas para o efeito. Contém ainda um ficheiro de configuração \emph{cmake} e os ficheiros relativos à biblioteca \emph{tinyxml2}.

Na pasta \textbf{Generator} residem os ficheiros relativos ao programa \emph{generator}. Contém também um ficheiro de configuração \emph{cmake}.

Na pasta \textbf{Generated\_Models} residem os ficheiros que contêm os pontos gerados para cada figura, criados pelo programa \emph{generator}.

Na pasta \textbf{Scenes} residem os ficheiros \emph{XML} que contêm as estruturas do sistemas solares desenvolvidos para esta fase do projeto. Construímos dois ficheiros deste tipo, um que contém os raios relativos de cada planeta e do sol baseados na realidade (grande disparidade entre as figuras criadas), e o outro que é mais elucidativo e percetível. 


\newpage

\section{Parser XML}
\label{sec:parser}

\subsection{Ficheiro}
\label{sec:ficheiro}

\subsection{Funcionamento}
\label{sec:funcionamento}


\newpage

\section{Estruturas de Dados}
\label{sec:estruturas}

\subsection{Tree}
\label{sec:tree}

\subsection{Figure}
\label{sec:figure}

\subsection{Color}
\label{sec:color}

\subsection{Scale}
\label{sec:scale}

\subsection{Rotation}
\label{sec:rotation}

\subsection{Translation}
\label{sec:translation}

\subsection{Point}
\label{sec:point}


\newpage

\section{\textit{Render Scene}}

\newpage

\section{\textit{Upgrades da Fase Anterior}}

Em relação à fase anterior houve alguns pontos que foram sujeitos a mudanças e otimizações pelo que estas serão relatadas nesta secção.

Uma das partes que foi sujeita a otimização foi o cálculo das coordenadas da câmera que na fase anterior continha demasiado código e algumas variáveis que não tinham nexo em ser usadas.
Desta forma decidimos que esta parte deveria ser reescrita utilizando uma abordagem mais objetiva. O resultado é descrito na seguinte figura.

\begin{figure}[H]
\centering
\includegraphics[scale=0.9]{camara.png}
\caption{Código referente às coordenadas da câmara.}
\label{img:camara}
\end{figure}

Não só a função que calcula as coordenadas para câmara mas também a própria função da câmara foi melhorada. Passou a poder mover-se consoante o ponto para o qual está a olhar. Desta forma possibilitou-nos a facilidade de podermos deslocarmo-nos para qualquer sítio que nos convenha. Tal alteração é descrita na seguinte figura.

\begin{figure}[H]
\centering
\includegraphics[scale=0.9]{mat_camara.png}
\caption{Função que contém a matriz da câmara.}
\label{img:mat_camara}
\end{figure}

!!!!!!!!!!!!!!!!!!!!!!!!!!!!!!!!!!!!!!!!! FALTA ACABAR !!!!!!!!!!!!!!!!!!!!!!!!!!!!!!!!!!!!!!!!!

FALAR DOS .H

\newpage

\section{Conclusão}
\label{sec:conclusao}

!!!!!!!!!!!!!!!!!!!!!!!!!!!!!!!!!!!!!!!!! POR FAZER !!!!!!!!!!!!!!!!!!!!!!!!!!!!!!!!!!!!!!!!!

Terminada a realização da fase 1 - primitivas gráficas, o grupo sente que realizou com sucesso  as 4 primitivas gráficas que compunham esta fase (plane, box, sphere e cone).  Assim, depois de uma revisão integral ao trabalho achamos que estamos preparados e num bom caminho para a realização de um bom projeto.
No geral não sentimos muitas dificuldades em nenhuma das primitivas gráficas pois também contamos com a ajuda da câmera que implementamos que permitiu uma melhor visualização do que estávamos a fazer bem ou mal.


\section{Bibliografia}
\label{sec:bibliografia}

!!!!!!!!!!!!!!!!!!!!!!!!!!!!!!!!!!!!!!!!! POR FAZER !!!!!!!!!!!!!!!!!!!!!!!!!!!!!!!!!!!!!!!!!

https://www.suapesquisa.com/astronomia/distancia\_sol\_planetas.htm

\end{document}
